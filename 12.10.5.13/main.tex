\documentclass[10pt]{article}
\usepackage{graphicx}
\usepackage[none]{hyphenat}
\usepackage{graphicx}
\usepackage{listings}
\usepackage[english]{babel}
\usepackage{siunitx}
\usepackage{graphicx}
\usepackage{caption} 
\usepackage{booktabs}
\usepackage{array}
\usepackage{amssymb} % for \because
\usepackage{amsmath}   % for having text in math mode
\usepackage{extarrows} % for Row operations arrows
\usepackage{listings}
\usepackage[utf8]{inputenc}
\lstset{
  frame=single,
  breaklines=true
}
\usepackage{hyperref}
  
%Following 2 lines were added to remove the blank page at the beginning
\usepackage{atbegshi}% http://ctan.org/pkg/atbegshi
\AtBeginDocument{\AtBeginShipoutNext{\AtBeginShipoutDiscard}}


%New macro definitions
\newcommand{\mydet}[1]{\ensuremath{\begin{vmatrix}#1\end{vmatrix}}}
\providecommand{\brak}[1]{\ensuremath{\left(#1\right)}}
\newcommand{\solution}{\noindent \textbf{Solution: }}
\newcommand{\myvec}[1]{\ensuremath{\begin{pmatrix}#1\end{pmatrix}}}
\providecommand{\norm}[1]{\left\lVert#1\right\rVert}
\providecommand{\abs}[1]{\left\vert#1\right\vert}
\let\vec\mathbf{}
\begin{document}

\begin{center}
\title{\textbf{Vector Algebra}}
\date{\vspace{-5ex}} %Not to print date automatically
\maketitle
\end{center}

\begin{enumerate}
\item\textbf{Problem statement :} The scalar product of the vector $\hat{i}+\hat{j}+\hat{k}$ with a unit vector along the sum of vectors $2\hat{i}+4\hat{j}-5\hat{k}$ and $\lambda\hat{i}+2\hat{j}+3\hat{k}$ is equal to one, Find the value of $\lambda$.
\\
\solution
Let
\begin{align}
\vec{a} =\myvec{1\\1\\1} , \vec{b}=\myvec{2\\4\\-5} , \vec{c}=\myvec{\lambda\\2\\3}\\
\vec{b}+\vec{c}=\myvec{2\\4\\-5}+\myvec{\lambda\\2\\3}=\myvec{2+\lambda\\6\\-2}
\end{align}
Let $\vec{r}$ be the unit vector along with $\vec{b}+\vec{c}$
\begin{align}
\hat{\vec{r}}=\frac{\vec{b}+\vec{c}}{\norm{\vec{b}+\vec{c}}}=\frac{\vec{b}+\vec{c}}{\sqrt{\brak{2+\lambda}^2+\brak{6}^2+\brak{-2}^2}}\\
\implies\hat{\vec{r}}=\frac{1}{\sqrt{\lambda^2+4\lambda+4}}
\implies\hat{\vec{r}}=\frac{1}{\sqrt{\lambda^2+4\lambda+4}}\myvec{2+\lambda\\6\\-2}
\end{align}
Given $\vec{a}^\top\brak{\vec{\hat{r}}} = 1$
\begin{align}
\myvec{1&1&1}\brak{{\sqrt{\lambda^2+4\lambda+4}}\myvec{2+\lambda\\6\\-2}}=1\\
\implies \myvec{1&1&1}\brak{\myvec{2+\lambda\\6\\-2}}={\sqrt{\lambda^2+4\lambda+4}}\\
\implies 2+\lambda+6-2 = {\sqrt{\lambda^2+4\lambda+4}} \\
\implies \lambda+6 = {\sqrt{\lambda^2+4\lambda+4}}\\
\end{align}
Squaring on both sides
\begin{align}
    \brak{\lambda+6}^2 = \brak{{\sqrt{\lambda^2+4\lambda+4}}}^2\\
    \lambda^2+12\brak{\lambda}+4 = \lambda^2+4\lambda+4\\
    8\brak{\lambda} = 8\\
    \lambda = 1
\end{align}
\end{enumerate}
\end{document}
